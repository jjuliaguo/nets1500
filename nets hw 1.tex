\documentclass{article}
\usepackage[letterpaper,top=0.5in,bottom=0.5in,left=1.2in,right=1.2in,includeheadfoot]{geometry}
\usepackage{amssymb,amsmath}
\usepackage{parskip}
\usepackage{fancyhdr}
\usepackage{tabu}
\usepackage{enumerate}
\usepackage{graphicx}
\usepackage{xcolor}
\newcommand{\solution}[1]{\item {#1} \newpage}
\newcommand{\multichoose}[2]{\left(\!\genfrac{(}{)}{0pt}{}{#1}{#2}\!\right)}
\renewcommand{\baselinestretch}{1.3}
% NEEDS CHANGING
\newcommand{\HomeworkNo}{6}
\pagestyle{fancy}
\fancyhead{}
\fancyfoot[L]{\MyName{}}
\fancyfoot[R]{Recitation time: \MyRecitation{} - Homework \HomeworkNo{}}
\renewcommand{\headrulewidth}{0pt}
\renewcommand{\frac}[2]{\dfrac{#1}{#2}}
\setlength{\headheight}{0pt}
\fancypagestyle{firstpage}{
\fancyhead{}
\fancyfoot{}
}
%% Please enter your name & recitation directly below
\newcommand{\MyName}{Julia Guo}
\newcommand{\MyRecitation}{201}
\newcommand{\CollaboratorNames}{Fimi Ma, Katherine Smith}
%% Do not modify the name below here
\newcommand{\MyCollaboratorCheck}[1]
{\ifnum\pdfstrcmp{#1}{GENERIC COLLABORATOR NAMES THAT NEEDS CHANGING}=0
\textcolor{red}{#1}
\else
#1
\fi}
\newcommand{\MyNameCheck}[1]
{\ifnum\pdfstrcmp{#1}{GENERIC NAME THAT NEEDS CHANGING}=0
\LARGE\textcolor{red}{#1}
\else
#1
\fi}
\newcommand{\MyRecitationCheck}[1]{\ifnum\pdfstrcmp{#1}{X}=0
\huge\textcolor{red}{#1}
\else
#1
\fi}
\newcommand{\PrintFirstHeader}{
CIS 1600 25S \vspace{5pt} \hfill {\Large\MyNameCheck{\MyName}}
{\LARGE{\textbf{Homework \HomeworkNo{}}}} \vspace{5pt} \hfill Recitation Number:
{\large\MyRecitationCheck{\MyRecitation}}\\
Collaborators: \MyCollaboratorCheck{\CollaboratorNames}
\rule{\textwidth}{0.4pt}
}
\begin{document}
\thispagestyle{firstpage}
\PrintFirstHeader{}
%
%
%
%% ONLY MODIFY BELOW HERE (Besides entering your name)
%
%
%
\begin{enumerate}[\bf Q1.]
\setlength\itemsep{1em}
% Question 1
\solution {
\begin{enumerate}
\item 0.106488
\item 0.104512
\item 1 - 0.104512 = 0.895488
\item 0.387464
\item 1/16
\item 0.03125
\item \(
\binom{5}{3} \cdot (0.5)^3 \cdot (0.5)^2
\)
\item \((0.5)^2\) = 0.25
\item 1- \((0.5)^5\) = 0.96875
\end{enumerate}
}
% Question 2
\solution {
We want to find the probability that both selected cards have a number value between 2 to 10 inclusive and at least one of them is red.

There's 52 total cards. Each suit has 9 valid cards. Valid refers to if the card has a number value between 2-10 inclusive. So the total number of valid cards is \(4 \cdot 6 = 36\)

We can find the total possible ways to pick any two cards from the deck (this is also the sample space):
\[
\binom{52}{2} = \frac{52 \times 51}{2} = 1326.
\]

Now, we count favorable outcomes (both cards are ranked 2-10 and at least one is red):

First, count the ways to pick two cards ranked 2-10. There are 36 valid cards, so:

\[
\binom{36}{2} = \frac{36 \times 35}{2} = 630.
\]

Then, we want to use complementary counting to address 'at least one is red.' So find the number of ways where both cards are black because we can subtract that from the number of ways to choose 2 cards from the valid ones. There are 18 valid black cards, so the number of ways to pick two black cards:

\[
\binom{18}{2} = \frac{18 \times 17}{2} = 153.
\]

Then actually use complementary counting for the number of ways where at least one card is red

Total number of ways to choose 2 cards from the valid cards - number of ways to pick 2 black cards (no red card)

\[
630 - 153 = 477.
\]

Now, we compute the probability by taking number of ways where at least one card is red divided by the total ways to pick any two cards from the deck:

\[
\frac{477}{1326}.
\]
}
% Question 3
\solution {
The probability that \( x, y, z \) are even =  \( \frac{1}{2} \) =  probability that \( x, y, z \) are odd. These are independent of each other.

Let \( a, b \) be arbitrary integers and \( a, b \in \mathbb{Z}^+ \).
We know from previous lectures (1.2) that 

- if \( a, b \) are both even, then their sum is even

- if \( a, b \) are both odd, then their sum is even

- WLOG if a is odd and b is even then the sum will be odd (we don't care about odd)

From 8.3, we know in order to check for pairwise independence, we should define:

- Let \( A \) = \( x + y \) is even

- Let \( B \) = \( y + z \) is even

- Let \( C \) = \( x + z \) is even

The sample space is all of the possible triplets \( (x, y, z) \) where \( x, y, z \) are either even or odd. We can calculate this as 8 (will see later on).

Let \( E \)= even 

Let \( O \) = odd

This will help us determine and visualize the possible outcomes

Since \( x, y, z \) can be either even or odd (2 outcomes). By MR, there are \( 2 \times 2 \times 2 = 8 \) possible outcomes because there are 3 variables:

\[
OEE, OEO, OOE, OOO, EEE, EEO, EOE, EOO
\]

Thus, the sample size is 8.

We know from lecture slides that for pairwise independence this must be true:

\[
P(A \cap B) = P(A) P(B)
\]

\[
P(A \cap B) = \frac{2}{8} = \frac{1}{4}
\]

Now, we calculate for \( P(A) \) and \( P(B) \):

\[
P(A) = \frac{4}{8} = \frac{1}{2}, \quad P(B) = \frac{4}{8} = \frac{1}{2}.
\]

Then, multiply them together:

\[
P(A) P(B) = \frac{1}{2} \times \frac{1}{2} = \frac{1}{4}.
\]

Since \( P(A \cap B) = P(A) P(B) \), we have just proved that the events \( A \) and \( B \) are pairwise independent. 
This also holds for \( A \) and \( C \), and for \( B \) and \( C \) because of WLOG since all the possible outcomes of \( P(A \cap B)\) are either even or odd. This is all divided by the total of 8. So, we've proved that \( A, B, C \) are pairwise independent.

Now, we can check for mutual independence. From lecture, we know that this requires:

\[
P(A \cap B \cap C) = P(A) P(B) P(C).
\]

From earlier, we know that:

\[
P(C) = \frac{4}{8} = \frac{1}{2}.
\]

\[
P(A) P(B) P(C) = \frac{1}{2} \times \frac{1}{2} \times \frac{1}{2} = \frac{1}{8}.
\]

Now, we can calculate \( P(A \cap B \cap C) \), which consists of cases where \( x, y, z \) are all even or all odd:

\[
P(A \cap B \cap C) = \frac{2}{8} = \frac{1}{4}.
\]

\( \frac{1}{4} \neq \frac{1}{8} \).
Therefore, it is not mutually independent.

We can conclude that \( A, B, C \) are only pairwise independent.

}
% Question 4
\solution {
The sample space for this problem is all ways to choose 8 souvenirs from the 24 available. David picks souvenirs from the 6 TAs. This can be written as:

\[
\binom{24}{8}
\]

We know that each choice is equally likely.

To find the probability that David took at least one souvenir from each TA, we consider two different cases that are exhaustive for this problem.

\textbf{Case 1: David takes one souvenir from 4 TAs and two souvenirs from 2 TAs}  

This means that out of the 6 TAs, David randomly picks 2 TAs to steal two souvenirs from, while the other 4 TAs will only have one souvenir stolen

The number of ways to pick 2 TAs (where David steals 2 souvenirs) out of 6 is:

  \[
  \binom{6}{2} = 15
  \]

From each of the two chosen TAs, David will pick 2 souvenirs out of 4:

  \[
  \binom{4}{2}^2
  \]

From each of the remaining 4 TAs, David will pick 1 souvenir out of 4:

  \[
  \binom{4}{1}^4
  \]

So, the total number of ways for this case is:

\[
\binom{6}{2} \cdot \binom{4}{2}^2 \cdot \binom{4}{1}^4
\]

\textbf{Case 2: David takes one souvenir from 5 TAs and three souvenirs from 1 TA}  

This means that out of the 6 TAs, David randomly picks 1 TA to steal three souvenirs from, while the other 5 TAs will only have one souvenir stolen

The number of ways to pick 1 TA (where David steals 3 souvenirs) out of 6 is:

  \[
  \binom{6}{1} = 6
  \]

From this chosen TA, David will pick 3 souvenirs out of 4:

  \[
  \binom{4}{3}
  \]

From each of the remaining 5 TAs, David will pick 1 souvenir out of 4:

  \[
  \binom{4}{1}^5
  \]

So, the total number of ways for this case is:

\[
6 \cdot \binom{4}{3} \cdot \binom{4}{1}^5
\]

Since these two cases cover all possible ways David could have taken at least one souvenir from each TA, we apply the AR:

\[
\binom{6}{2} \cdot \binom{4}{2}^2 \cdot \binom{4}{1}^4 + 6 \cdot \binom{4}{3} \cdot \binom{4}{1}^5
\]

Now, to compute the probability, we divide by the total:

\[
\frac{\binom{6}{2} \cdot \binom{4}{2}^2 \cdot \binom{4}{1}^4 + 6 \cdot \binom{4}{3} \cdot \binom{4}{1}^5}{\binom{24}{8}}
\]

Thus, this gives the probability that David took at least one souvenir from each TA.
}
% Question 5
\solution {
The tic-tac-toe grid is \( 3 \times 3 \), so there are 9 spaces. Each space can either be filled with an X or an O, so the total number of possible grids is:

\[
2^9 = 512.
\]

The condition for Max to win is that at least one \( 2 \times 2 \) grid must be completely filled with X’s. There are 4 possible (exhaustive) ways to form a \( 2 \times 2 \) grid in a \( 3 \times 3 \) grid.

Each square has an equal probability of being an X or an O:

\[
P(\text{each square is X}) = \frac{1}{2}.
\]

Since a \( 2 \times 2 \) grid consists of 4 squares, and the probabilities are independent, the probability that a \( 2 \times 2 \) grid is all X’s is:

\[
\left(\frac{1}{2}\right)^4 = \frac{1}{16}.
\]

Lets define \( X_1, X_2, X_3, X_4 \) = the events where each of the four \( 2 \times 2 \) grids is filled with X’s. 

We need to find:

\[
P(X_1 \cup X_2 \cup X_3 \cup X_4),
\]

because this represents the probability that at least one of these events occurs.

We can use the inclusion-exclusion principle that we learned from previous modules.

\[
\begin{aligned}
P(X_1 \cup X_2 \cup X_3 \cup X_4) &= P(X_1) + P(X_2) + P(X_3) + P(X_4) \\
&\quad - P(X_1 \cap X_2) - P(X_1 \cap X_3) - P(X_2 \cap X_4) \\
&\quad - P(X_3 \cap X_4) \\
&\quad + P(X_1 \cap X_2 \cap X_3) \\
&\quad + P(X_1 \cap X_3 \cap X_4) + P(X_2 \cap X_3 \cap X_4) \\
&\quad - P(X_1 \cap X_2 \cap X_3 \cap X_4).
\end{aligned}
\]

Calculate the individual probabilities. Each event represents 4 squares being filled with X’s. Since there are 9 squares in total, and each square is independently either X or O, the probability of any one of these events happening is:

\[
P(X_1) = P(X_2) = P(X_3) = P(X_4) = \left(\frac{1}{2}\right)^4 = \frac{1}{16}.
\]

Then, calculate the probability of two overlapping \( 2 \times 2 \) grids being filled. 
If two grids overlap, that means that they share 2 squares, so the probability that both grids are all X’s is:

\[
P(X_1 \cap X_2) = P(X_1 \cap X_3) = P(X_1 \cap X_4) = P(X_2 \cap X_3) = P(X_2 \cap X_4) = P(X_3 \cap X_4) = \left(\frac{1}{2}\right)^6 = \frac{1}{64}.
\]

For three overlapping \( 2 \times 2 \) grids, this means that the grids share 5 squares, leaving 4 squares that are independently determined. The probability that three of these grids are completely filled with X’s is:

\[
P(X_1 \cap X_2 \cap X_3) = P(X_1 \cap X_2 \cap X_4) = P(X_1 \cap X_3 \cap X_4) = P(X_2 \cap X_3 \cap X_4) = \left(\frac{1}{2}\right)^7 = \frac{1}{128}.
\]

For all four \( 2 \times 2 \) grids being filled, this means that all 9 squares must be X’s, so:

\[
P(X_1 \cap X_2 \cap X_3 \cap X_4) = \left(\frac{1}{2}\right)^9 = \frac{1}{512}.
\]

Now, we can apply the inclusion-exclusion formula:

\[
P(X_1 \cup X_2 \cup X_3 \cup X_4) = 4 \cdot \frac{1}{16} - 6 \cdot \frac{1}{64} + 3 \cdot \frac{1}{128} - \frac{1}{512}.
\]
So the probability that Max will win is = 
\[
\frac{4 \cdot 2^{-4} - 4 \cdot 2^{-3} + 3 \cdot 2^{-2} - 2^{-9}}{2^9}
\]

}
% Question 6
\solution {
\begin{enumerate}
\item 
We can refer to the Monty Hall problem's probabilities from 8.4.

The sample space consists of all possible ways Megan can pick a door and then decide whether to switch after Manny opens a Michigan door.

Let A, B, C be the 3 doors

Each outcome is represented as a triplet in the form:

\[
(D_{MO}, D_{ME}, D_M)
\]

where:

- \( D_{MO} \) = the door that hides the Moldova tickets.

- \( D_{ME} \) = the door that Megan chooses initially.

- \( D_M \) = the door that Manny opens (which always contains Michigan tickets).

Since Megan can initially pick any of the three doors, and Manny always opens a door revealing Michigan, we list all possible outcomes:

AAB, AAC, ABC, ACB, BAC, BBA, BBC, BCA, CAB, CBA, CCA, CCB
(From 8.5 lecture slides)

- Megan does not switch doors:  
  \[
  P(AAB) = P(AAC) = P(BBA) = P(BBC) = P(CCA) = P(CCB) = \frac{1}{18}.
  \]
- Megan switches doors:  
  \[
  P(ABC) = P(ACB) = P(BAC) = P(BCA) = P(CAB) = P(CBA) = \frac{1}{9}.
  \]

We can determine this because if she does not switch, then \( D_{MO} = D_{ME} \) (her initial choice was already the Moldova ticket door).

We can break down how Megan wins Moldova tickets in two cases:  

1. By switching her door choice.
2. By staying with her original choice.

\textbf{Case 1: If Megan wins Moldova tickets after switching her door choice}

The probability that Megan initially picks Michigan (B or C) is:

  \[
  P(\text{Megan picks Michigan}) = \frac{2}{3}. \text{because of 1/9 times 6}
  \]

If Megan switches doors, she must have flipped heads on the coin. The probability of flipping heads = probability of success:

  \[
  P(\text{Heads}) = p.
  \]

Since these choices are independent, by MR, the probability that Megan wins Moldova tickets by switching is:

  \[
  P(\text{Moldova} \mid \text{Switching}) = \frac{2}{3} \times p.
  \]

\textbf{Case 2: If Megan wins Moldova tickets after sticking with her original door choice}

- The probability that Megan initially picks Moldova (A) is:

  \[
  P(\text{Megan picks Moldova}) = \frac{1}{3} \text{ because 1 - } \frac{2}{3} \text{=} \frac{1}{3} \text{ or also } \frac{1}{18}\cdot6 = \frac{1}{3}
  \]

If Megan does not switch, this means she flipped tails = probability of failure, which has probability:

  \[
  P(\text{Tails}) = 1 - p.
  \]

Since these choices are independent, by MR, the probability that Megan wins Moldova tickets by staying is:

  \[
  P(\text{Moldova} \mid \text{Not Switching}) = \frac{1}{3} \times (1 - p).
  \]

Since these two cases are mutually exclusive and exhaustive, we apply the Law of Total Probability:

\[
P(\text{Moldova}) = P(\text{Moldova} \mid \text{Switching}) + P(\text{Moldova} \mid \text{Not Switching}).
\]

\[
P(\text{Moldova}) = \frac{2}{3} p + \frac{1}{3} (1 - p).
\]

\item
We can use complementary counting.

So total - undesirable probability = \(1 -\) probability that Megan has less than a 50\% chance of going to Moldova.

The probability that Megan has less than a 50\% chance of going to Moldova is 0.5 because it accounts for probabilities from 0\% to 49\% inclusive. 

Thus, 
\[
1 - 0.5 = 0.5.
\]

Therefore, the smallest \( p \) that gives Megan at least a 50\% chance of going to Moldova is 0.5.
\item 
The probability that Megan did not switch doors, given that Mohit ended up in Michigan is a conditional probability problem, so we can use the formula and concepts mentioned from 8.5:

\[
P(\text{E} \mid \text{U}) = \frac{P(\text{E} \cap \text{U})}{P(\text{U})}.
\]

Apply it to this problem:

\[
P(\text{Not Switching} \mid \text{Mohit in Michigan}) = \frac{P(\text{Not Switching} \cap \text{Mohit in Michigan})}{P(\text{Mohit in Michigan})}.
\]

From the problem, we know that probability of heads = probability of success =

\[
p = \frac{1}{5}.
\]

This means the probability of flipping tails (not switching doors) = probability of failure is:

\[
1 - p = \frac{4}{5}.
\]

Mohit can end up in Michigan in two ways:

1. Megan does not switch and originally picks Michigan.

2. Megan switches and originally picks Moldova.

\textbf{Case 1: Megan does not switch and originally picks Michigan.}

The probability that Megan initially chooses Michigan is:

  \[
  P(\text{Michigan is initial pick}) = \frac{2}{3} \text{ because 2 of the 3 doors lead to Michigan tickets}
  \]

If she does not switch, this means she flipped tails, which has probability:

  \[
  P(\text{Tails}) = 1 - p = \frac{4}{5}.
  \]

Since these choices are independent, use MR:

  \[
  P(\text{Not Switching} \cap \text{Mohit in Michigan}) = \frac{2}{3} \times \frac{4}{5} = \frac{8}{15}.
  \]

\textbf{Case 2: Megan switches and originally picks Moldova.}

The probability that Megan initially chooses Moldova is:

  \[
  P(\text{Moldova is initial pick}) = \frac{1}{3} \text{ because 1 of the 3 doors lead to Moldova tickets}
  \]

If she switches, she must have originally picked Moldova, and she must have flipped heads in order to switch.

The probability of flipping heads is:

  \[
  P(\text{Heads}) = p = \frac{1}{5}.
  \]

Since these choices are independent, by MR:

  \[
  P(\text{Switching} \cap \text{Mohit in Michigan}) = \frac{1}{3} \times \frac{1}{5} = \frac{1}{15}.
  \]

So the total probability that Mohit ends up in Michigan is:

\[
P(\text{Mohit in Michigan}) = P(\text{Not Switching} \cap \text{Mohit in Michigan}) + P(\text{Switching} \cap \text{Mohit in Michigan}).
\]

\[
P(\text{Mohit in Michigan}) = \frac{8}{15} + \frac{1}{15}.
\]

\[
P(\text{Mohit in Michigan}) = \frac{9}{15}.
\]

We need to find the probability that Megan did not switch given that Mohit ended up in Michigan. So we can go back to the earlier formula:

\[
P(\text{Not Switching} \mid \text{Mohit in Michigan}) = \frac{P(\text{Not Switching} \cap \text{Mohit in Michigan})}{P(\text{Mohit in Michigan})}.
\]

\[
P(\text{Not Switching} \mid \text{Mohit in Michigan}) = \frac{\frac{8}{15}}{\frac{9}{15}}.
\]
\[= \frac{8}{9}.
\]

Thus, the probability that Megan did not switch doors given that Mohit ended up in Michigan is \( \frac{8}{9} \).


\end{enumerate}
}
% Question 7
\solution {
Assuming that there is at least 1 TA in every location of the 3.

The sample space is every possible way that Maggie can make her 2 selections of locations while keeping the order in which she selects first. Each selection is independent from one another.

\textbf{Case 1}: Maggie clicks on \( a \) first

There is a probability of \( \frac{a}{a+b+c} \) that she chooses \( a \).
After putting it back, there is a probability of \( \frac{b+c}{a+b+c} \) for her to choose a location that is not \( a \).

Thus, the probability that she clicked on 2 different locations is:
\[
\frac{a}{a+b+c} \cdot \frac{b+c}{a+b+c}.
\]

\textbf{Case 2}: Maggie clicks on \( b \) first

There is a probability of \( \frac{b}{a+b+c} \) that she chooses \( b \).
After putting it back, there is a probability of \( \frac{a+c}{a+b+c} \) for her to choose a location that is not \( b \).

Thus, the probability that she clicked on 2 different locations is:
\[
\frac{b}{a+b+c} \cdot \frac{a+c}{a+b+c}.
\]

\textbf{Case 3}: Maggie clicks on \( c \) first

There is a probability of \( \frac{c}{a+b+c} \) that she chooses \( c \).
After putting it back, there is a probability of \( \frac{a+b}{a+b+c} \) for her to choose a location that is not \( c \).

Thus, the probability that she clicked on 2 different locations is:
\[
\frac{c}{a+b+c} \cdot \frac{a+b}{a+b+c}.
\]

By the AR, the final probability is:
\[
\frac{a}{a+b+c} \cdot \frac{b+c}{a+b+c} + \frac{b}{a+b+c} \cdot \frac{a+c}{a+b+c} + \frac{c}{a+b+c} \cdot \frac{a+b}{a+b+c}.
\]

} 
% Question 8
\solution {
Let A = the event where Professor randomly checks one of Group A’s plane tickets and it's to Boston.

Let B = the event where the Boston plane icket was transferred from Group B to Group A.

Let C = the event where the Chicago plane ticket was transferred from Group B to Group A.

We want to find the probability that Professor Tannen accidentally transferred the Boston ticket from Group B to Group A given that he randomly checks one of Group A’s tickets and finds it to be a ticket to Boston. In other words, we we are trying to find \( P(B \mid A) \).

From the conditional probability formula in 8.5, we know:

\[
P(B \mid A) = \frac{P(A \cap B)}{P(A)}
\]

We also know:

\[
P(A \cap B) = P(A \mid B) \cdot P(B)
\]

We can substitute in and get:

\[
P(B \mid A) = \frac{P(A \mid B) \cdot P(B)}{P(A)}
\]

Now, we must find the individual probabilities that make up this equation.

The probability that Professor Tannen checks a Boston ticket given that a Boston ticket was transferred:

\[
P(A \mid B) = \frac{2}{6}
\]

This is because the total number of tickets in Group B is 6. After transferring a Boston ticket, there are 2 Boston tickets left in Group A. The probability of randomly checking one of them is \( \frac{2}{6} \).

Then, the probability that a Boston ticket was transferred:
\[
P(B) = \frac{3}{4}
\]
This is because Group B initially has 4 tickets, and 3 of them are Boston tickets. Since the transfer is random, the probability of transferring a Boston ticket is \( \frac{3}{4} \).

Then, the probability that Professor Tannen checks a Boston ticket given that a Chicago ticket was transferred:

\[
P(A \mid C) = \frac{1}{6}
\]

If a Chicago ticket was transferred, then Group A has only 1 Boston ticket out of 6 total tickets remaining. The probability of checking it is \( \frac{1}{6} \).

Lastly, the probability that a Chicago ticket was transferred:

\[
P(C) = \frac{1}{4}
\]

Since Group B initially had 4 tickets, only 1 of them was a Chicago ticket. The probability of transferring it is \( \frac{1}{4} \).

Since B and C are disjoint events, the total probability of checking a Boston ticket is:

\[
P(A) = P(A \cap B) + P(A \cap C)
\]

Using MR:

\[
P(A) = P(A \mid B) \cdot P(B) + P(A \mid C) \cdot P(C)
\]

\[
= \frac{2}{6} \cdot \frac{3}{4} + \frac{1}{6} \cdot \frac{1}{4}
\]

\[
= \frac{6}{24} + \frac{1}{24} = \frac{7}{24}
\]

Now, we can calculate \( P(B \mid A) \):

\[
P(B \mid A) = \frac{P(A \mid B) \cdot P(B)}{P(A)}
\]

\[
= \frac{\frac{2}{6} \cdot \frac{3}{4}}{\frac{7}{24}}
\]

\[
= \frac{6}{24} \div \frac{7}{24}
\]

\[
= \frac{6}{7}
\]

Thus, the probability that Professor Tannen accidentally transferred a Boston ticket from Group B to Group A given that he checked a Boston ticket is \( \frac{6}{7} \)


}
\end{enumerate}
%
%
%
%% ONLY MODIFY ABOVE HERE
\end{document}\\\\\\